\section{План главы}

Данная глава знакомит нас со средствами лексического и синтаксического анализа,
которые входят в дистрибутив Objective CAML. Обычно, синтаксический анализ
следует за лексическим. В первой части мы узнаем о простом инструменте
лексического анализа из модуля \texttt{Genlex}. После этого ознакомимся с
формализмом рациональных выражений и тем самым рассмотрим более детально
определение множества лексических единиц. А так же проиллюстрируем их реализацию
в модуле \texttt{Str} и инструменте \texttt{ocamllex}. Во второй части мы
определим грамматику и рассмотрим правила создания фраз языка. После этого
рассмотрим два анализа фраз: восходящий и нисходящий. Они будут
проиллюстрированы использованием \texttt{Stream} и \texttt{ocamlyacc}. В
приведенных примерах используется контекстно–независимая грамматика. Здесь мы
узнаем как реализовать контекстный анализ при помощи Stream. В третьей части мы
вернемся к интерпретатору BASIC (см. стр \cite{??}) и при помощи
\texttt{ocamllex} и \texttt{ocamlyacc} добавим лексические и синтаксические
функции анализа языка.
