\section{Введение}

Определение и реализация средств лексического и синтаксического анализа являлись
важным доменом исследования в информатике. Эта работа привела к созданию
генераторов лексического и синтаксического анализа \texttt{lex} и
\texttt{yacc}. Команды \texttt{camllex camlyacc}, которые мы представим в этой
главе, являются их достойными наследниками. Два указанных инструмента стали
de--facto стандартными, однако существуют другие средства, как например потоки
или регулярные выражения из библиотеки \texttt{str}, которые могут быть
достаточны для простых случаев, там где не нужен мощный анализ.

Необходимость подобных инструментов особенно чувствовалась в таких доменах, как
компиляция языков программирования. Однако и другие программы могут с успехом
использовать данные средства: базы данных, позволяющие определять запросы или
электронная таблица, где содержимое ячейки можно определить как результат
какой--нибудь формулы. Проще говоря, любая программа, в которой взаимодействие с
пользователем осуществляется при помощи языка, использует лексический и
синтаксический анализ.

Возьмём простой случай. ASCII формат часто используется для хранения данных,
будь то конфигурационный системный файл или данные табличного файла. Здесь, для
использования данных, необходим лексический и синтаксический анализ.

Обобщая, скажем что лексический и синтаксический анализ преобразует линейный
поток символов в данные с более богатой структурой: последовательность слов,
структура записи, абстрактное синтаксическое дерево программы и т.д.

У каждого языка есть словарный состав (лексика) и грамматика, которая описывает
каким образом эти составные объединяются (синтаксис). Для того, чтобы машина или
программа могли корректно обрабатывать язык, этот язык должен иметь точные
лексические и синтаксические правила. У машины нет \enq{тонкого чувства} для
того чтобы правильно оценить двусмысленность натуральных языков. По этой
причине, к машине необходимо обращаться в соответствии с чёткими правилами, в
которых нет исключений. В соответствии с этим, понятия лексики и семантики
получили формальные определения, которые будут кратко представлены в данной
главе.