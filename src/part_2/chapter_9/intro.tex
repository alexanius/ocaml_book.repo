\section {Введение}

Средства анализа программ предоставляют дополнительную информацию, по сравнению
с компилятором или компоновщиком. Часть этих средств реализуют статический
анализ. То есть они анализируют (в виде исходного кода или синтаксического
дерева) и определяют различные свойства кода, такие как зависимости между
модулями или возможные исключения. Другие средства проводят динамический анализ,
то есть изучают программы во время ее выполнения. Такие средства полезны, в
случае когда необходимо знать сколько раз была вызвана определенная функция,
какие аргументы были ей переданы или время, затраченное на выполнение части кода
программы. Такие средства анализа могут быть интерактивными, как например
отладчик программ. В этом случае выполнение программы изменяется, для того чтобы
реагировать на команды пользователя, который может установить контрольные точки
чтобы просмотреть значения или перезапустить программу с новыми аргументами.

В дистрибутив Objective CAML входят подобные средства. Некоторые из них имеют не
часто встречающиеся характеристики, связанные со статической типизацией. Такая
типизация гарантирует, что во время выполнения кода, не произойдет ошибок типа и
используется компилятором для того чтобы генерировать эффективный код небольшого
размера. Часть информации о типе созданных значений теряется и из–за этой
особенности нельзя просмотреть аргументы полиморфных функций.
