\section {Память программы}

Машинные коды программы --- это последовательность инструкций, изменяющих
значения в памяти. Память в основном состоит из следующиъ элементов:

\begin{itemize}
	\item регистры процессора,

	\item стек

	\item сегмент данных (область статического выделения памяти)

	\item куча (область динамического выделения памяти)
\end{itemize}

Только стек ит динамическая область выделения памяти могут менять свой размер
во время работы программы. В зависимости от используемого языка, можно
осуществлять управление данными классами памяти. В то время как программные
инструкции (код) обычно находятся в статической памяти, динамическое связывание
(см. стр. \cite{??}) использует динамическую память.
