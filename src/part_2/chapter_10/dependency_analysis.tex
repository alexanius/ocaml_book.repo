\section {Анализ зависимостей}
\label{sec:dependency_analysis}

Вычисление зависимостей между файлами интерфейса и реализации имеет двойную 
цель. Первая --- получить глобальное видение зависимостей между модулями? 
Вторая --- использовать данную информацию для того чтобы компилировать лишь 
необходимый минимум при изменении некоторых файлов. 

Команда \code{ocamldep} анализирует файлы \code{.ml} и \code{.mli} и выводит 
зависимости файлов в формате Makefile \footnote{Файлы \code{Makefile} 
используются командой \code{make} чтобы обновлять при необходимости файлы или 
программы}.

Зависимости появляются при глобальной декларации других модулей, либо используя 
синтаксис с точкой (пример: \code{M1.f}), либо открывая модуль (пример: 
\code{open M1}).

Предположим, что имеются файлы \code{dp.ml}:

\begin{lstlisting}[language=OCaml]
let print_vect v = 
    for i = 0 to Array.length v do 
      Printf.printf "%f " v.(i)
    done;
    print_newline();;
\end{lstlisting}

и \code{d1.ml}:

\begin{lstlisting}[language=OCaml]
let init n e = 
  let v = Array.create 4 3.14 in 
    Dp.print_vect v;
    v;;
\end{lstlisting}

При вызове команды \code{ocamldep} с указанием файлов, получим следующий 
результат:

\begin{lstlisting}[language=Bash]
$ ocamldep dp.ml d1.ml array.ml array.mli printf.ml printf.mli
dp.cmo: array.cmi printf.cmi 
dp.cmx: array.cmx printf.cmx 
d1.cmo: array.cmi dp.cmo 
d1.cmx: array.cmx dp.cmx 
array.cmo: array.cmi 
array.cmx: array.cmi 
printf.cmo: printf.cmi 
printf.cmx: printf.cmi 
\end{lstlisting}

Зависимости вычисляются для байт--код и нативных компиляторов. Поясним 
полученный результат: компиляция файла \code{dp.cmo} зависит от файлов 
\code{array.cmi} и \code{printf.cmi}. Файлы с расширением \code{.cmi} зависят 
от файлов с таким же именем, но с расширением .mli. Аналогичное правило 
распространяется на файлы \code{.ml} с \code{.cmo} и \code{.cmx}.

Объектные файлы дистирбитува отсутствуют в списке зависимостей. В 
действительности, если команда \code{ocamldep} не найдёт их в текущем каталоге, 
то они будут найдены в каталоге библиотек дистрибутива и выдаст следующий 
результат:

\begin{lstlisting}[language=Bash]
$ ocamldep dp.ml d1.ml
d1.cmo: dp.cmo 
d1.cmx: dp.cmx 
\end{lstlisting}

Для того, чтобы указать командe \code{ocamldep} дополнительные каталоги для 
поиска файлов, используйте опцию \code{-I каталог}.
