\section{Резюме}
\label{sec:summary_10}

В данной главе мы ознакомились с различными вспомогательными инструментами 
разработки программ, которые входят в дистрибутив Objective CAML.

Первое из этих средств реализует статический анализ и определяет зависимости 
между множеством элементов компиляции. Эта информация затем интегрируются в файл 
\code{Makefile}, что позволяет компилировать только необходимые файлы. То есть, 
если вы изменили какой--то файл исходник, то необходимо перекомпилировать лишь 
сам файл и зависящие от него файлы, а не всю программы.

Другие средства предоставляют информацию о выполнении программы. Интерактивная 
среда предлагает трассировку выполнения, но, как мы видели, полиморфизм 
устанавливает достаточно серьёзные ограничения на исследуемые значения. Нам 
\enq{видны} лишь мономорфные глобальные объявления, а так же моморфные 
параметры функции. С помощью этого мы все такие можем отслеживать выполнение 
рекурсивных функций. 

И последний инструмент является традиционным в окружении Unix, i.e. отладчик и 
профайлер. При помощи отладчика можно пошагово выполнять программу, а профайлер 
выдаёт информацию о производительности програмы. Оба этих средства могут быть 
использованы в полной мере лишь в среде Unix.
