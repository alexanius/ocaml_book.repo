\section {Введение}
\label{sec:intro_8}

В наборе с каждым языком программирования идут программы, которые могут быть 
использованы программистом --- они называются библиотеками. Количество и 
качество библиотек является одним из основных критериев удобства использования 
языка. Библиотеки можно разбить на два типа. В первый тип библиотек 
предоставляет типы и функции частого использования и которые могут быть 
определены языком. Второй --- предоставляющий возможности, которые не могут 
быть определены языком. Иначе говоря, при помощи первого типа библиотек, мы 
избавляемся от переопределений таких вещей как стеки, очередь, и т.д., а второй 
тип расширяет возможности языка.

Большое количество библиотек поставляется в дистрибутиве Objective CAML. Они 
распространяются в виде скомпилированных файлов. Однако, любознательный читатель 
может найти исходные файлы библиотек в дистрибутиве исходников языка.

Библиотеки Objective CAML организованы по модулям, которые в свою очередь 
являются элементами компиляции. Каждый из них содержит глобальные определения 
типов, исключений и значений, которые могут быть использованы в программе. В 
данной главе, мы не будем рассматривать построение подобных модулей, а лишь 
использование существующих. В \ref{??} главе мы вернёмся к концепции модуля 
(логический элемент) и элемента компиляции и опишем язык описания модулей в 
Objective CAML. А в \ref{??} мы обсудим включение кода написанного на другом 
языке в библиотеки Objective CAML, в частности интеграция кода на C в Objective 
CAML.

В дистрибутив Objective CAML входит предустановленная библиотека (модуль 
\code{Pervasives}), набор базовых модулей, называемых стандартной библиотекой и 
много других библиотек, добавляющих дополнительные возможности. Некоторые 
библиотеки лишь упоминаются в данной главе или они описываются в следующих 
главах.
