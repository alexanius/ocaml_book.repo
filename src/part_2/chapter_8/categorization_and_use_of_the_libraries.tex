\section {Классификация и использование библиотек}
\label{sec:categorization_and_use_of_the_libraries}

Библиотеки дистрибутива Objective CAML могут быть разделены на три части. В 
первой содержатся предустановленные глобальные объявления. Вторая, так 
называемая стандартная библиотека, имеет достаточно высокий уровень 
стабильности. Она разделена на 4 части:

\begin{itemize}
	\item структуры данных

	\item input/output

	\item системный интерфейс

	\item синтаксический и лексический анализ 
\end{itemize}

В третьей части библиотек находятся модули расширяющие возможности языка, как 
например библиотека \code{Graphics} (см. главу 
\ref{chpt:the_graphics_interface}). В этой части мы можем найти библиотеки для 
обработки регулярных выражений (\code{Str}), точной арифметики (\code{Num}), 
системных вызовов Unix (\code{Unix}), потоков (\code{Threads}) и динамической 
загрузки byte--code (\code{Dynlink}).

Операции ввода/вывода и системный интерфейс стандартной библиотеки совместимы с 
различными операционными системами: Unix, Windows, MacOS. Не все библиотеки 
третьей группы обладают такой особенностью. Также существует множество библиотек 
поставляемых независимо от дистрибутива Objective CAML.

\paragraph{Использование}

Для того, чтобы использовать модуль или библиотеку в программе, используют 
синтаксис с точкой, указав имя модуля, а затем имя объекта. К примеру, если 
необходим объект \code{f} из модуля \code{Name}, то мы пишем \code{Name.f}. 
Чтобы постоянно не добавлять префикс из имени модуля, можно открыть модуль и 
вызывать \code{f} напрямую.

\paragraph{Синтаксис}

\begin{lstlisting}[language=OCaml]
open Name
\end{lstlisting}

С этого момента, все объявления модуля \code{Name} будут рассматриваются как 
глобальные в текущем окружении. Если какое-то объявление имеет одно и то же имя 
в двух библиотеках, то имя последней загруженной библиотеки перекрывает 
предыдущее объявление. Чтобы вызвать первое объявление, необходимо использовать 
синтаксис с точкой.