\section{Введение}
\label{sec:intro_13}

В этой главе представлены два приложения, которые призваны продемонстрировать
пользу от многих подходов к программированию, продемонстрированных ранее.

Первым приложением мы создадим библиотеку графических элементов, \texttt{Awi} 
(Application Window Interface) - Оконный интерфейс приложения. Следующая 
библиотека будет для простого конвертера из франков в Евро. Компоненты 
библиотеки реагируют на ввод со стороны пользователя вызовом обработчиков 
событий. Хотя алгоритмически это простое приложение, оно показывает 
преимущества 
использования замыканий чтобы структурировать взаимодействие между 
компонентами. 
В самом деле, различные обработчики событий имеют общие значения, которые им 
видны через их окружение. Для оценки архитектуры Awi необходимо знать базовую 
графическую библиотеку (\ref{??}).

Второе приложение ищет кратчайший путь в ориентированном графе. Оно использует 
алгоритм Дейкстры, который вычисляет кратчайший путь из исходного узла до всех 
остальных. Механизм кеширования, реализованный через использование таблицы 
слабых указателей (weak pointers) (см. \ref{??}), используется для ускорения 
поиска. Сборщик мусора может удалить элементы таблицы в любое время, но они 
могут быть посчитаны заново при необходимости. Для выбора исходного и 
конечного узла при поиске пути, визуализированный граф использует  простой 
компонент кнопки, взятый из библиотеки \texttt{Awi}. Далее мы сравним 
эффективность работы алгоритма с кешированием и без него. Для облегчения замера 
временных различий между двумя версиями алгоритма, файл с описанием графа и 
источника с конечным узлом подаётся в качестве аргумента поисковому алгоритму. 
В конце небольшой графический интерфейс будет добавлен к программе поиска.
