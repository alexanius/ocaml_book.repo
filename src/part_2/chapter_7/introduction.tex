\section {Введение}

Для того, чтобы текст программы превратился в исполняемый модуль, необходимо
выполнить несколько операций. Эти операции сгруппированы в процессе компиляции.
В ходе этого процесса, строится абстрактное синтаксическое дерево (как для
интерпретатора Basic, \ref{??}), затем оно превращается в последовательность
инструкций для реального процессора или для виртуальной машины. В последнем
случае, для того чтобы выполнить программу, необходим интерпретатор инструкций
виртуальной машины. В каждом случае, результат компиляции должен быть связан с
библиотекой времени выполнения, входящей в дистрибутив. Она может меняться в
зависимости от процессора и операционной системы. В неё входят примитивные
функции (такие как операции над числами, системный интерфейс) и менеджер
памяти.

В Objective CAML существует два компилятора. Первый из них --- это компилятор
для виртуальной машины, в результате которого мы получаем байт--код. Второй
компилятор создаёт программу, состоящую из машинного кода,
выполняемого реальным процессором, как Intel, Motorola, SPARC, HP-PA,
Power-PC или Alpha. Компилятор байт--кода отдаёт предпочтение переносимости
кода, тогда как компилятор в машинный код увеличивает скорость выполнения.
Интерактивный интерпретатор, который мы видели в первой части, использует
байт--код. Каждая введённая фраза компилируется и затем выполняется в окружении
символов, определённых в течении интерактивной сессии.