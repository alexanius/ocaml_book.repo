\section{Автономный исполняемый файл}
\label{sec:standalone_executables}

Автономный исполняемый файл (standalone) есть исполняемый файл, независящий от
дистрибутива Objective CAML на используемой машине. С одной стороны, бинарный
файл упрощает распространение программ, с другой стороны, он не зависит от
расположения дистрибутива Objective CAML на компьютере.

Компилятор машинного кода всегда генерирует автономный исполняемый файл. Тогда
как без опции \texttt{-custom}, байт--код компилятор создаст исполняемый файл,
который нуждается в интерпретаторе {\it ocamlrun}. Пусть файл
\texttt{example.ml} содержит следующее:

\begin{lstlisting}[language=OCaml]
let f x = x + 1;;
print_int (f 18);;
print_newline();;
\end{lstlisting}

Следующая команда создаёт файл \texttt{example.exe}:

\begin{lstlisting}[language=OCaml]
ocamlc -o example.exe example.ml
\end{lstlisting}

Полученный файл, размером около 8Kb, может быть выполнен при помощи следующей
команды:

\begin{lstlisting}[language=OCaml]
$ ocamlrun example.exe
19
\end{lstlisting}

Интерпретатор выполняет команды виртуальной машины Zinc, хранящиеся в файле
\texttt{example.exe}

Для операционной системы Unix, в первой строке такого файла находится путь к
байт--код интерпретатору, например:

\begin{lstlisting}[language=OCaml]
#!/usr/local/bin/ocamlrun
\end{lstlisting}

При помощи этой строки, файл может быть напрямую запущен интерпретатором. Таким
образом, выполнение файла \texttt{example.exe} запускает файл, путь к которому
находится в первой строке. Если интерпретатор не найдёт по этому пути
программу--интерпретатор, сообщение Command not found будет выведено на экран и
выполнение остановится.

Та же компиляция с опцией \texttt{-custom} создаст автономный файл с именем
\texttt{exauto.exe}:

\begin{lstlisting}[language=OCaml]
ocamlc -custom -o exauto.exe example.ml
\end{lstlisting}

Теперь размер файла около 85Kb, так как он содержит не только байт--код, но и
интерпретатор. Этот файл может быть самостоятельно запущен или скомпилирован на
другую машину (той же архитектуры и той же операционной системы) для запуска.
