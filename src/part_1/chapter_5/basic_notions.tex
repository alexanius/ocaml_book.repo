\section{Основные понятия}
\label{sec:basic_notions}

Программирование графических интерфейсов изначально связано с развитием
аппаратных средств, в частности мониторов и графических карт. Для того чтобы
изображение имело хорошее качество, необходимо чтобы оно регулярно и часто
обновлялось (перерисовывалось), как в кино. Для рисования на экране существует
два основных метода: либо используя список видимых сегментов, при этом только
нужная часть изображения рисуется на экране, либо рисуя все точки экрана 
(растровый экран). На обычных компьютерах используется последний способ.

Растровый экран можно рассматривать как прямоугольник точек, называемые пиксель 
от английского {\it picture element}. Они являются базовыми элементами 
изображения. Высота и ширина экрана называется разрешением основного растра. 
Размер этого растра зависит от памяти занимаемой пикселем. Для черно--белого 
изображения пиксель может занимать один бит. Для цветных или серых изображений 
размер пикселя зависит от числа оттенков ассоциированных пикселю. Для растра 
$320 \times 640$ пикселей по 256 цветов в каждом, нам понадобится 8 битов на 
каждый пиксель, соответственно размер видео памяти должен быть $480 * 640$ 
байтов = 307200 байтов $\approx$ 300 КБ. Это разрешения до сих пор используется 
некоторыми программами в MS-DOS.

В следующем списке приведены базовые операции над растром из библиотеки
\texttt{Graphics}:

\begin{itemize}
	\item раскраска пикселя
	
	\item рисование пикселя
	
	\item рисование контуров: прямоугольник, эллипс
	
	\item заливка замкнутых контуров: прямоугольник, эллипс, многоугольник
	
	\item рисование текста: как растрового так и векторного
	
	\item изменение и перемещение частей изображения 
\end{itemize}

Каждая из этих операций использует координаты точки изображения для задания
места рисования. Некоторые характеристики графических операций образуют
графический контекст: толщина линии, соединение линий, выбор шрифта и его
размер, стиль заливки. Графическая операция всегда выполняется в определённом
контексте и её результат зависит от этого контекста. Графический контекст
библиотеки \texttt{Graphics} содержит лишь текущие точку, цвет, шрифт и его
размер.