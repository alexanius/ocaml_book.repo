\section{Основные понятия}
\label{sec:basic_notions}

Программирование графических интерфейсов изначально связано с развитием
аппаратных средств, в частности мониторов и графических карт. Для того чтобы
изображение имело хорошее качество, необходимо чтобы оно регулярно и часто
обновлялось (перерисовывалось), как в кино. Для рисования на экране существует
два основных метода: либо используя список видимых сегментов, при этом только
нужная часть изображения рисуется на экране, либо рисуя все точки экрана (экран
bitmap). На обычных компьютерах используется последний способ.

Экран bitmap можно рассматривать как прямоугольник точек, называемые пиксель от
английского picture element. Они являются базовыми элементами изображения.
Высота и ширина экрана называется разрешением основного bitmap. Размер этого
bitmap зависит от памяти занимаемой пикселем. Для черно–белого изображения
пиксел может занимать один бит. Для цветных или серых изображений размер пиксела
зависит от числа оттенков ассоциированных пикселю. Для bitmap 320x640 пикселей
по 256 цветов в каждом, нам понадобится 8 битов на каждый пиксел, соответственно
размер видео–памяти должен быть 480*640 байтов = 307200 байтов $\approx$ 300 KB.
Это разрешения до сих пор используется некоторыми программами в MS-DOS.

В следующем списке приведены базовые операции над bitmap из библиотеки
\texttt{Graphics}:

\begin{itemize}
	\item закрашивание пикселя
	
	\item рисование сегмента
	
	\item рисование формы: прямоугольник, эллипс
	
	\item заливка замкнутой формы: прямоугольник, эллипс, многоугольник
	
	\item рисование текста
	
	\item манипуляция и перемещение части изображения 
\end{itemize}

Каждая из этих операций использует координаты точки изображения для задания
места рисования. Некоторые характеристики графических операций образуют
графический контекст: толщина линии, соединение линий, выбор шрифта и его
размер, стиль заливки. Графическая операция всегда выполняется в определённом
контексте и её результат зависит от этого контекста. Графический контекст
библиотеки \texttt{Graphics} содержит лишь текущие точку, цвет, шрифт и его
размер.