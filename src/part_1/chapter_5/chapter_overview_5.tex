\section{План главы}
\label{sec:chapter_overview_5}

В первом разделе мы объясним как пользоваться этой библиотекой на различных
системах. Во втором изучим основы графического программирования: точка, рисунок,
заполнение, цвет, растровые изображения (bitmap). Для того чтобы 
проиллюстрировать эти концепции, в третьем разделе опишем и реализуем функции 
рисования \enq{блоков} (boxes). В четвёртом разделе увидим анимацию графических 
объектов и их взаимодействие с фоном экрана или другими анимационными объектами. 
В пятом разделе представлен стиль программирования событиями, а так же скелет 
любого графического приложения. И наконец, в последнем разделе мы воспользуемся 
библиотекой \texttt{Graphics} для реализации интерфейса калькулятора 
приведённого в разделе \ref{sec:calculator_with_memory}.