\section{Графический вывод}
\label{sec:graphical_display}

При графическом выводе следующие элементы являются базовыми: начальная точка
(reference point), графический контекст, цвет, изображение, закраска замкнутых
фигур, тексты и растровые изображения.

\subsection{Ориентир и графический контекст}
\label{subsec:reference_point_and_graphical_context}

Библиотека \texttt{Graphics} управляет единственным и главным окном. Координаты
ориентира окна могут меняться от нижней левой точки (0,0) до верхнего правого
угла. Вот основные операции над этим окном:

\begin{itemize}
	\item \texttt{open\_graph}, тип которой \texttt{string -> unit}, открывает
окно

	\item \texttt{close\_graph}, тип которой \texttt{unit -> unit}, закрывает

	\item \texttt{clear\_graph}, тип которой \texttt{unit -> unit}, и очищает 
\end{itemize}

Размер окна определяется функциями \texttt{size\_x} и \texttt{size\_y}.

Строка, которую мы передаём функции \texttt{open\_graph}, зависит от графической
оболочки машины где запускается программа, то есть она платформо зависимая. Если
строка пустая, то полученное окно будет иметь свойства по умолчанию. Мы можем
явно указать размер окна; в X-Window строка \texttt{`` 200x300''} создаст окно
размером 200 пискелей по ширине и 300 по высоте. Заметьте, пробел --- первый
символ строки \texttt{`` 200x300''}, обязателен.

В графическом контексте есть несколько компонентов которые можно
просмотреть/изменить:

\begin{tabular}{rl}
	текущая точка: & \texttt{current\_point : unit -> int * int} \\
	 & moveto : \texttt{int -> int -> unit} \\
	текущий цвет : & \texttt{set\_color : color -> unit} \\
	ширина текущей линии : & \texttt{set\_line\_width : int -> unit} \\
	текущий шрифт : & \texttt{set\_font : string -> unit} \\
	размер этого шрифта : & \texttt{set\_text\_size : int -> unit} \\
\end{tabular}

\subsection{Цвета}
\label{subsec:colors}

Цвет представлен тремя байтами, каждый из которых хранит яркость основного
цвета в модели RGB (red, green, blue) в интервале от 0 до 255. Функция rgb (с
типом \texttt{int -> int -> int -> color}) создаёт цвет с тремя указанными
значениями. Если все три значения равны, то мы получим серый цвет более или
менее светлый, в зависимости от значений. Чёрный соответствует минимуму яркости
для каждой компоненты цвета rgb(0,0,0), белый цвет --- rgb(255,255,255). Есть
некоторые предопределённые цвета: \texttt{black, white, red, green, blue,
yellow, cyan, magenta}.

Переменные foreground и background соответствуют текущему цвету и цвету фона.
При стирании экрана, осуществляется заполнение цветом фона.

Цвет (значение типа \texttt{color}) это на самом деле целое число, которое мы
можем разбить на три части (\texttt{from\_rgb}) или инвертировать
(\texttt{inv\_color}).

\begin{lstlisting}[language=OCaml]
(* color == R * 256 * 256  +  G * 256  +  B  *)
# let from_rgb (c : Graphics.color)  =
   let r = c / 65536  and  g = c / 256 mod 256  and  b = c mod 256
   in (r,g,b);;
val from_rgb : Graphics.color -> int * int * int = <fun>
# let inv_color (c : Graphics.color) =
    let (r,g,b) = from_rgb c
    in Graphics.rgb (255-r) (255-g) (255-b);;
val inv_color : Graphics.color -> Graphics.color = <fun>
\end{lstlisting}

Функция \texttt{point\_color}, типа \texttt{int -> int -> color}, возвращает
цвет точки, координаты которой указаны на входе.

\subsection{Рисунок и заливка}
\label{subsec:drawing_and_filling}

Функция рисования выводит линию на экран, при этом для толщины и цвета
используются текущие значения. Функция заливки закрашивает замкнутую форму
текущим цветом. Различные функции рисования и заливки приведены в таблице
\ref{tbl:drawing_and_filling_functions}

Функции рисования и заливки
\begin{table}[hl]
	\begin{center}
	\caption{\label{tbl:drawing_and_filling_functions} Функции рисования и
заливки}
	\begin{tabular}{|l|l|r|}
		\hline
		рисунок & заполнение & тип \\
		\hline
		plot & & int -> int -> unit \\
		\hline
		lineto & & int -> int -> unit \\
		\hline
		& fill\_rect & int -> int -> int -> int -> unit \\
		\hline
		& fill\_poly & ( int * int) array -> unit \\
		\hline
		draw\_arc & fill\_arc & int -> int -> int -> int -> int -> unit \\
		\hline
		draw\_ellipse & fill\_ellipse & int -> int -> int -> int -> unit \\
		\hline
		draw\_circle & fill\_circle & int -> int -> int -> unit \\
		\hline
	\end{tabular}
	\end{center}
\end{table}

Заметьте, что функция \texttt{lineto} имеет побочный эффект: она изменяет
положение текущей точки.

\subsubsection{Рисование многоугольников}