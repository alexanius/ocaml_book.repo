\section{Введение}
\label{sec:introduction_5}

В этой главе представлена библиотека Graphics, она входит является частью
дистрибутива языка Objective CAML. Эта библиотека одинаково работает в
графических интерфейсах основных платформ: Windows, MacOS, Unix с оболочкой
X-Windows. С помощью Graphics мы можем реализовать графические рисунки с
текстом, картинками, управлять различными базовыми событиями как нажатие на
кнопку мыши или клавиатуры.

Модель программирования графических рисунков --- \enq{модель художника}:
последний нарисованный слой стирает предыдущий. Это императивная модель,
графическое окно в ней представляется как вектор пикселей, которые физически
изменяются различными графическими функциями. Взаимодействие с мышью и
клавиатурой подчиняется модели программирования событиями: главная функция
программы это бесконечный цикл, в котором мы ожидаем действие пользователя.

Несмотря на простоту библиотеки Graphics, она вполне достаточна для введения
концепций программирования графических интерфейсов с одной стороны и с другой
стороны содержит базовые элементы простые в использовании программистом, при
помощи которых мы можем реализовать более сложные интерфейсы.