\section{Резюме}
\label{sec:summary_5}

 В этой главе мы представили основы графического программирования и 
программирование событиями, используя библиотеку \code{Graphics} дистрибутива 
Objective CAML. Сначала мы рассмотрели базовые элементы (цвет, рисунок, заливка, 
текст и растр), затем изучили анимацию этих элементов. После введения 
механизма обработки событий библиотеки \code{Graphics}, мы увидели общий метод 
управления действиями пользователя используя программирование событиями. Для 
того чтобы улучшить интерактивность и предложить разработчику интерактивные 
графические компоненты была разработана библиотека, упрощающая создание 
графических интерфейсов, \code{Awi}. Это библиотека была использована при 
написании интерфейса императивного калькулятора.
