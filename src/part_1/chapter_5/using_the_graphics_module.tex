\section{Использование библиотеки \texttt{Graphics}}
\label{sec:using_the_graphics_module}

Использование библиотеки зависит от операционной системы и способа компиляции.
Здесь мы рассмотрим только программы используемые в интерактивном цикле
Objective CAML. В операционных систем Windows и MacOS интерактивное рабочее
окружение само загружает библиотеку. Для систем Unix необходимо создать новый
toplevel \footnote{в оригинале на английском, прим. пер.}, который зависит от
того где находится библиотека X11. Если она находится в одном из каталогов по
умолчанию, где ищутся библиотеки C, то командная строка будет выглядеть так:

\begin{lstlisting}[language=Bash]
ocamlmktop -custom -o mytoplevel graphics.cma -cclib -lX11
\end{lstlisting}

Этим мы создаем команду mytoplevel с включённой библиотекой X11. Запуск этой
команды осуществляется так:

\begin{lstlisting}[language=Bash]
./mytoplevel
\end{lstlisting}

Если же библиотека расположена в другом каталоге, как в Linux, необходимо его
явно указать:

\begin{lstlisting}[language=Bash]
ocamlmktop -custom -o montoplevel graphics.cma -cclib \
-L/usr/X11/lib -cclib -lX11
\end{lstlisting}

В этом примере файл libX11.a ищется в каталоге /usr/X11/lib.

Более подробный пример команды ocamlmktop приведён в главе \ref{??}.