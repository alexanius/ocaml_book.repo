\section{Введение}
\label{sec:intro_6}

Преимущество одного языка программирования над другим заключается в простоте 
разработки качественных программ и лёгком сопровождении программного 
обеспечения. Первая часть книги, посвящённая представлению языка Objective 
CAML, вполне естественно завершится реализацией нескольких программ.

В первой программе мы реализуем несколько функций запрашивающих информацию из 
базы данных. Наше внимание будет акцентировано на функциональном стиле 
программирования и использование списков. Таким образом пользователь будет иметь 
набор функций для формулирования и выполнения запросов прямо в языке Objective 
CAML. В этом примере мы покажем разработчику как он может с лёгкостью 
предоставить набор функций необходимых пользователю.

Вторая программа это интерпретатор BASIC \footnote{сокращение Beginner's All 
purpose Symbolic Instruction Code }. Напомним, что подобные императивные языки 
принесли немалый успех первым микрокомпьютерам. Двадцать лет спустя, реализация 
таких языков является простой задачей. Несмотря на то что BASIC императивный 
язык, для написания интерпретатора мы воспользуемся функциональной частью 
Objective CAML, в частности для вычисления инструкций. Однако, для лексического 
и синтаксического анализа мы используем физически изменяемую структуру данных.

Третья программа --- всем известная игра Minesweeper, которая входит в 
стандартный дистрибутив Windows. Цель игры --- найти все спрятанные мины, 
исследуя рядом расположенные ячейки. Для реализации мы воспользуемся 
императивной частью языка, так как поле игры представлено в виде матрицы, 
которая будет изменятся после каждого хода игрока. Мы, также используем модуль 
\code{Graphics} для реализации интерфейса игры и обработки событий. Вычисление 
автоматически открывающихся ячеек будет сделано в функциональном стиле.

Данная программа использует модуль {\code{Graphics} описанный в главе 
\ref{chpt:the_graphics_interface} и несколько функций из модулей \code{Random} и 
\code{Sys} из главы \ref{??}.