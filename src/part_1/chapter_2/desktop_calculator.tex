\section{Калькулятор}
\label{sec:desktop_calculator}

Для того чтобы понять как организуются программы на Objective CAML, необходимо
самим реализовать такую программу. Напишем калькулятор манипулирующий целыми
числами при помощи 4 стандартных операций.

Для начала, определим тип \texttt{key}, для представления кнопки калькулятора.
Всего таких кнопок будет 15: по одной для каждой операции, для каждой цифры и
для кнопки равно.

\begin{lstlisting}[language=OCaml]
# type key = Plus | Minus | Times | Div | Equals | Digit of int ;;
\end{lstlisting}

Заметим, что цифровые кнопки сгруппированы под одним конструктором
\texttt{Digit} с аргументом целого типа. Таким образом, некоторые значения типа
\texttt{key} не являются на самом деле кнопкой. Например, (\texttt{Digit 32})
есть значение типа \texttt{key}, но не представляет ни одну кнопку калькулятора.

Напишем функцию \texttt{valid} которая проверяет входной аргумент на
принадлежность к типу \texttt{key}. Тип функции \texttt{key->bool}.

На первом этапе определим функцию проверяющую что число принадлежит интервалу
0-9. Объявим её с именем \texttt{is\_digit}.

\begin{lstlisting}[language=OCaml]
# let est_chiffre = function x -> (x>=0) && (x<=9) ;;
val est_chiffre : int -> bool = <fun>
\end{lstlisting}

Теперь функция \texttt{valid} фильтрующая свой аргумент тип которого
\texttt{key}:

\begin{lstlisting}[language=OCaml]
let valide tch = match tch with
     Chiffre n -> est_chiffre n
   | _ -> true ;;
val valide : touche -> bool = <fun>
\end{lstlisting}

Первый фильтр применяется в случае если входной аргумент построен конструктором
\texttt{Digit}, в этом случае он проверяется функцией \texttt{is\_digit}. Второй
фильтр применяется в любом другом случае. Напомним, что благодаря фильтру, тип
фильтруемого значения обязательно будет \texttt{key}.

Перед тем как начать реализовывать алгоритм калькулятора, определим модель
формально описывающую реакцию на нажатие кнопок. Мы подразумеваем что
калькулятор располагает памятью в которой хранится результат последней операции,
последняя нажатая кнопка, последний использованный оператор и число выведенное
на экран. Все эти 4 значения назовем состоянием калькулятора, оно изменяется
каждый раз при нажатии на одну из кнопок. Это изменение называется переходом, а
теория управляющая подобным механизмом есть теория автоматов.

Состояние представлено следующим типом:

\begin{lstlisting}[language=OCaml]
type etat = {
   dce : int;    (* dernier calcul effectué     *)
   dta : touche; (* dernière touche actionnée   *)
   doa : touche; (* dernier opérateur actionné  *)
   vaf : int     (* valeur affichée             *)
 } ;;
\end{lstlisting}

В таблице \ref{tbl:transitions_for_3+21*2=} приведён пример состояний
калькулятора.

\begin{table}[hl]
\begin{center}
	\caption{Переходы 3+21*2=}
	\begin{tabular}{|c|c|c|}
	\hline
	 & Состояние & Значение \\
	\hline
	 & $(0, =, =, 0)$ & 3 \\
	\hline
	$\to$ & $(0, 3, =, 3)$ & + \\
	\hline
	$\to$ & $(3, +, +, 3)$ & 2 \\
	\hline
	$\to$ & $(3, 2, +, 2)$ & 1 \\
	\hline
	$\to$ & $(3, 1, +, 21)$ & * \\
	\hline
	$\to$ & $(24, *, *, 24)$ & 2 \\
	\hline
	$\to$ & $(24, 2, *, 2)$ & = \\
	\hline
	$\to$ & $(48, =, =, 48)$ & \\
	\hline
	\end{tabular}
\end{center}
	\label{tbl:transitions_for_3+21*2=}
\end{table}

Нам нужна функция \texttt{evaluate} с тремя аргументами: двумя операндами и
оператором, она возвращает результат операции. Для этого функция фильтрует
входной аргумент типа \texttt{key}:

\begin{lstlisting}[language=OCaml]
# let evalue x y tch = match tch with
     Plus  -> x + y
   | Moins -> x - y
   | Fois  -> x * y
   | Par   -> x / y
   | Egal  -> y
   | Chiffre _ -> failwith "evalue : no op";;
val evalue : int -> int -> touche -> int = <fun>
\end{lstlisting}

Дадим определение переходам, перечисляя все возможные случаи. Предположим, что
текущее состояние есть квадри--плет \texttt{(a, b, A, d)}:

\begin{itemize}
	\item кнопка с цифрой x нажата, есть два возможных случая:

	\begin{itemize}
		\item предыдущая нажатая кнопка тоже цифра, значит пользователь
продолжает	набирать число и необходимо добавить \texttt{x} к значению
выводимому на экран	то есть изменить его на \texttt{d * 10 + x}. Новое состояние
будет:	\texttt{(a,(Digit x), A, d * 10 + x)}

		\item предыдущая нажатая кнопка не цифра, тогда мы только начинаем
писать	новое число. Новое состояние: \texttt{(a, (Digit x), A, x)}
	\end{itemize}

	\item кнопка с оператором B была нажата, значит второй операнд был полностью
введён и теперь мы хотим \enq{что-то} сделать с двумя операндами. Для этого
сохранена последняя операция (\texttt{А}). Новое состояние: \texttt{(A d,B,B,a A
d)}
\end{itemize}

Для того чтобы написать функцию \texttt{transition}, достаточно дословно
перевести в Objective CAML предыдущее описание при помощи сопоставления входного
аргумента. Кнопку с двумя значениями обработаем локальной функцией
\texttt{digit\_transition} при помощи сопоставления.

\begin{lstlisting}[language=OCaml]
# let transition et tou =
   let transition_chiffre n = function
       Chiffre _  ->  { et with dta=tou; vaf=et.vaf*10+n }
     |  _         ->  { et with dta=tou; vaf=n }
   in
      match tou with
          Chiffre p  ->  transition_chiffre p et.dta
        |  _         ->  let res = evalue et.dce et.vaf et.doa
                         in { dce=res; dta=tou; doa=tou; vaf=res }  ;;
val transition : etat -> touche -> etat = <fun>
\end{lstlisting}

Эта функция из \texttt{state} и \texttt{key} считает новое состояние
\texttt{state}.

Протестируем теперь программу:

\begin{lstlisting}[language=OCaml]
# let etat_initial = { dce=0; dta=Egal; doa=Egal; vaf=0 } ;;
val etat_initial : etat = {dce=0; dta=Egal; doa=Egal; vaf=0}
# let etat2 = transition etat_initial (Chiffre 3) ;;
val etat2 : etat = {dce=0; dta=Chiffre 3; doa=Egal; vaf=3}
# let etat3 = transition etat2 Plus ;;
val etat3 : etat = {dce=3; dta=Plus; doa=Plus; vaf=3}
# let etat4 = transition etat3 (Chiffre 2) ;;
val etat4 : etat = {dce=3; dta=Chiffre 2; doa=Plus; vaf=2}
# let etat5 = transition etat4 (Chiffre 1) ;;
val etat5 : etat = {dce=3; dta=Chiffre 1; doa=Plus; vaf=21}
# let etat6 = transition etat5 Fois ;;
val etat6 : etat = {dce=24; dta=Fois; doa=Fois; vaf=24}
# let etat7 = transition etat6 (Chiffre 2) ;;
val etat7 : etat = {dce=24; dta=Chiffre 2; doa=Fois; vaf=2}
# let etat8 = transition etat7 Egal ;;
val etat8 : etat = {dce=48; dta=Egal; doa=Egal; vaf=48}
\end{lstlisting}

Мы можем сделать тоже самое, передав список переходов.

\begin{lstlisting}[language=OCaml]
# let transition_liste et lt = List.fold_left transition et lt ;;
val transition_liste : etat -> touche list -> etat = <fun>
# let exemple = [ Chiffre 3; Plus; Chiffre 2; Chiffre 1; Fois; Chiffre 2; Egal ]
 in transition_liste etat_initial exemple ;;
- : etat = {dce=48; dta=Egal; doa=Egal; vaf=48}
\end{lstlisting}



