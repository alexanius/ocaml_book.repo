\section{План главы}

В данной главе продолжается представление базовых элементов Objective CAML
затронутых в предыдущей главе, но теперь мы заинтересуемся императивными
конструкциями. Глава разбита на 5 разделов. Первый, наиболее важный, раскрывает
различные физически изменяемые структуры данных и описывает их представление в
памяти. Во втором разделе кратко излагаются базовые операции ввода/вывода.
Третий знакомит с новыми итеративными структурами контроля. В четвёртом разделе
рассказывается об особенностях выполнения императивных программ, в частности о
порядке вычисления аргументов функции. В последнем разделе мы переделаем
калькулятор предыдущей главы в калькулятор с памятью.