\section{Введение}

В отличие от функционального программирования, где значение вычисляется
посредством применения функции к аргументам, не заботясь о том как это
происходит, императивное программирование ближе к машинному представлению, так
как оно вводит понятие состояния памяти, которое изменяется под воздействием
программы. Каждое такое воздействие называется инструкцией и императивная
программа есть набор упорядоченных инструкций. Состояние памяти может быть
изменено при выполнении каждой инструкции. Операции ввода/вывода можно
рассматривать как изменение оперативной памяти, видео памяти или файлов.

Подобный стиль программирования напрямую происходит от программирования на
ассемблере. Мы встречаем этот стиль в языках программирования первого поколения
({\it Fortran, C, Pascal}, etc). Следующие элементы Objective CAML соответствуют
приведённой модели:

\begin{itemize}
	\item физически изменяемые \footnote{в оригинальном издании (французском)
используется выражение физически изменяемые, тогда как в английском переводе
лишь изменяемые} структуры данных, такие как массив или запись с
изменяемыми (mutable) полями

	\item операции ввода/вывода

	\item структуры контроля выполнения программы как цикл и исключения.
\end{itemize}

Некоторые алгоритмы реализуются проще таким стилем программирования. Примером
может послужить произведение двух матриц. Несмотря на то что эту операцию можно
реализовать чисто функциональным способом, при этом списки заменят массивы, это
не будет ни эффективнее, ни естественней по отношению к императивному стилю.

Интерес интеграции императивной модели в функциональный язык состоит в написании
при необходимости подобных алгоритмов в этом стиле программирования. Два главных
недостатка императивного программирования по отношению к функциональному это:

\begin{itemize}
	\item усложнение системы типов языка и отклонение (rejecting) некоторых
программ, которые бы без этого рассматривались бы как \enq{корректные},

	\item необходимость учитывать состоянием памяти и порядок вычисления. 
\end{itemize}

Однако, при соблюдении некоторых правил написания программ, выбор стиля
программирования предоставляет большие возможности в написании алгоритмов, что
является главной целью языков программирования. К тому же, программа написанная
в стиле близкому к стилю алгоритма, имеет больше шансов быть корректной (или по
крайней мере быстрее реализована).

По этим причинам в Objective CAML имеются типы данных, значения которых
физически изменяемые, структуры контроля выполнения программ и библиотека
ввода/вывода в императивном стиле.
