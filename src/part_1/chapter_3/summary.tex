\section{Резюме}
\label{sec:summary_3}

В этой главе вы видели применение стилей императивного программирования
(физически изменяемые значения, ввод–вывод, структуры итеративного контроля) в
функциональном языке. Только \texttt{mutable} значения, такие как строки,
векторы и записи с изменяемыми полями могут быть физически изменены. Другие
значения не могут меняться после их создания. Таким образом мы имеем значения
\texttt{read--only} для функциональной части и значения \texttt{read--write} для
императивной.
