\section{Ввод/вывод}
\label{sec:input-output}

Функции в/в вычисляют значение (чаше всего типа \texttt{unit}), однако в ходе
вычисления, они изменяют состояние устройств в/в: изменение буфера клавиатуры,
перерисовка экрана, запись в файл и так далее. Для каналов ввода и вывода
определены два типа: \texttt{in\_channel} и \texttt{out\_channel}. При
обнаружении конца файла возбуждается исключение \texttt{End\_of\_file}.
Следующие константы соответствуют стандартным каналам ввода, вывода и ошибок
{\it a la} Unix: \texttt{stdin}, \texttt{stdout} и \texttt{stderr}.

\subsection{Каналы}
\label{subsec:channels}

Функции ввода/вывода стандартной библиотеки Objective CAML, манипулируют
каналами типа \texttt{in\_channel} и \texttt{out\_channel} соответственно. Для
создания подобных каналов используется функция:

\begin{lstlisting}[language=OCaml]
# open_in;;
- : string -> in_channel = <fun>
# open_out;;
- : string -> out_channel = <fun>
\end{lstlisting}

\texttt{open\_in} открывает файл \footnote{с правом на чтение при
необходимости}, если он не существует, возбуждается исключение
\texttt{Sys\_error}.

\texttt{open\_out} создаёт указанный файл если такой не существует, если же он
существует то его содержимое уничтожается.

\begin{lstlisting}[language=OCaml]
# let ic = open_in "koala";;
val ic : in_channel = <abstr>
# let oc = open_out "koala";;
val oc : out_channel = <abstr>
\end{lstlisting}

Функции закрытия каналов:

\begin{lstlisting}[language=OCaml]
# close_in ;;
- : in_channel -> unit = <fun>
# close_out ;;
- : out_channel -> unit = <fun>
\end{lstlisting}

\subsection{Чтение/запись}
\label{subsec:reading_and_Writing}

Стандартные функции чтения/записи:

\begin{lstlisting}[language=OCaml]
# input_line ;;
- : in_channel -> string = <fun>
# input ;;
- : in_channel -> string -> int -> int -> int = <fun>
# output ;;
- : out_channel -> string -> int -> int -> unit = <fun>
\end{lstlisting}

\begin{itemize}
	\item \texttt{input\_line ic}: читает из входного канала \texttt{ic} символы
до обнаружения возврата каретки или конца файла и возвращает в форме строки (не
включая возврат каретки).

	\item \texttt{input ic s p l}: считывает \texttt{l} символов из канала
\texttt{ic} и помешает их в строку \texttt{c} со смешением \texttt{p}. Функция
возвращает число прочитанных символов.

	\item \texttt{output oc s p l}: записывает в выходной канал \texttt{oc}
часть строки \texttt{s} начиная с символа p длинной \texttt{l}.
\end{itemize}

Следующие функции чтения (записи) из (в) стандартного входа:

\begin{lstlisting}[language=OCaml]
# read_line ;;
- : unit -> string = <fun>
# print_string ;;
- : string -> unit = <fun>
# print_newline ;;
- : unit -> unit = <fun>
\end{lstlisting}

Другие значения простых типов данных могут быть прочтены/записаны, значения
таких типов могут быть переведены в тип список символов.

\subsubsection{Локальное объявление и порядок выполнения}

Попробуем вывести на экран выражение формы \texttt{let x=e1 in e2}. Зная, что в
общем случае локальная переменная \texttt{x} может быть использована в
\texttt{e2}, выражение \texttt{e1} вычислено первым и только затем \texttt{e2}.
Если эти оба выражения есть императивные функции с побочным эффектом результат
которых (), мы выполнили их в правильном порядке. В частности, так как нам
известен тип возвращаемого результата \texttt{e1}: константа () типа
\texttt{unit}, мы получим последовательный вывод на экран используя
сопоставление с образцом ().

\begin{lstlisting}[language=OCaml]
# let () = print_string "and one," in
 let () = print_string " and two," in
 let () = print_string " and three" in
    print_string " zero";;
and one, and two, and three zero- : unit = ()
\end{lstlisting}

\subsection{Пример Больше/Меньше}
\label{subsec:Example_Higher_Lower}

В следующем примере мы приводим игру Больше/Меньше, состоящую в поиске числа,
после каждого ответа программа выводит на экран сообщение в соответствии с тем
что предложенное число больше или меньше загаданного.

\begin{lstlisting}[language=OCaml]
# let rec hilo n =
     let () = print_string "type a number: " in
     let i = read_int ()
     in
       if i = n then
         let () = print_string "BRAVO" in
         let () = print_newline ()
         in print_newline ()
       else
         let () =
           if i < n then
             let () = print_string "Higher"
             in print_newline ()
           else
             let () = print_string "Lower"
             in print_newline ()
         in hilo n ;;
val hilo : int -> unit = <fun>
\end{lstlisting}

Результат запуска

\begin{lstlisting}[language=OCaml]
# hilo 64;;
type a number: 88
Lower
type a number: 44
Higher
type a number: 64
BRAVO

- : unit = ()
\end{lstlisting}
