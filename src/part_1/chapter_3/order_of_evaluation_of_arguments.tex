\section{Порядок вычисления аргументов}
\label{sec:order_of_evaluation_of_arguments}

В функциональном языке программирования порядок вычисления аргументов не имеет
значения. Из–за того что нет ни изменения памяти, ни приостановки вычисления,
расчёт одного аргумента не влияет на вычисление другого. Objective CAML
поддерживает физически изменяемые значения и исключения, поэтому пренебречь
порядком вычисления аргументов нельзя. Следующий пример специфичен для Objective
CAML 3.12.1 ОС Linux на платформе Intel:

\begin{lstlisting}[language=OCaml]
# let new_print_string s = print_string s; String.length s ;;
val new_print_string : string -> int = <fun>
# (+) (new_print_string "Hello ") (new_print_string "World!") ;;
World!Hello - : int = 12
\end{lstlisting}

По выводу на экран мы видим что вторая строка печатается после первой.

Таков же результат для исключений:

\begin{lstlisting}[language=OCaml]
# try (failwith "function") (failwith "argument") with Failure s -> s;;
Warning 20: this argument will not be used by the function.
- : string = "argument"
\end{lstlisting}

Если необходимо указать порядок вычисления аргументов, необходимо использовать
локальные декларации, форсируя таким образом порядок перед вызовом функции.
Предыдущий пример может быть переписан следующим способом:

\begin{lstlisting}[language=OCaml]
# let e1 = (new_print_string "Hello ")
 in let e2 = (new_print_string "World!")
 in (+) e1 e2 ;;
Hello World!- : int = 12
\end{lstlisting}

В Objective CAML порядок вычисления не указан, на сегодняшний день все
реализации caml делают это слева направо. Однако рассчитывать на это может быть
рискованно, в случае если в будущем язык будет реализован иначе.

Это вечный сюжет дебатов при концепции языка. Нужно ли специально не указывать
некоторые особенности языка и предложить программистам не пользоваться ими,
иначе они рискуют получить разные результаты для разных компиляторов. Или же
необходимо их указать и, следовательно, разрешить программистам ими
пользоваться, что усложнит компилятор и сделает невозможным некоторые
оптимизации?
