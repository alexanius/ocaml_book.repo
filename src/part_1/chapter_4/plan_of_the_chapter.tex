\section{План главы}
\label{sec:plan_of_the_chapter_4}

В этой главе мы сравним функциональную и императивную модель Objective CAML по
критерию контроля выполнения программы и представления значений в памяти. Смесь
обоих стилей позволяет конструировать новые структуры данных. Это будет
рассмотрено в первом разделе. Во втором разделе мы обсудим выбор между
композицией функций (composition of functions) или последовательности
(sequencing) с одной стороны и разделение (sharing) или копирование значений с
другой. Третий раздел выявляет интерес к смешиванию двух стилей для создания
функциональных изменяемых данных (mutable functional data), что позволит
создавать не полностью вычисленные (evaluated) данные. В четвёртом разделе
рассмотрены streams, потенциально бесконечные потоки данных и их интеграция,
посредством сопоставления с образцом.
