\section{Резюме}
\label{sec:summary_4}

В этой главе мы сравнили функциональный и императивный стили программирования.
Основные различия состоят в контроле выполнения (неявный в функциональном и
явный в императивном стиле) и представлении в памяти данных (явное разделение
или копирования в императивном стиле, не имеющее такой важности в
функциональном). Реализация алгоритмов в функциональном и императивном стилях
подразумевает эти различия. Выбор между обоими стилями на самом деле приводит к
их одновременному использованию. Это позволяет явно выразить представление
замыкания, оптимизировать критические части программы и создать изменяемые
функциональные данные. Физическое изменение значений в окружении замыкания
помогает нам лучше понять что такое функциональное значение. Одновременное
использование обоих стилей предоставляет мощные средства для реализации. Мы
воспользовались этим при создании потенциально бесконечных данных. 